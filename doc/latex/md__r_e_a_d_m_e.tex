@ mainpage Hexapode Documentation \section*{Hexapode }

Mini-\/projet hexapode

\subsection*{Cahier des charges}

Nous sommes en charge du développement du firmware d'un robot hexapode, basé sur Free\-R\-T\-O\-S. Le robot sera piloté via une liaison série sans fil type Zig\-Bee.

\paragraph*{Mécanique}

Le robot hexapode dispose de 6 pattes en 2 rangées, chaque patte est articulée en trois points via 3 servomoteurs classiques de modélisme. À cela s'ajoute une tourelle 2 axes sur l'avant équipée d'un télémètre infrarouge et d'une caméra C\-M\-U\-C\-A\-M2. De plus, chaque patte possède un contact T\-O\-R au bout du pied.

\paragraph*{Électronique}

La cible est un L\-P\-C17xx auquel sont connectés directement tous les éléments du robot \-: contacts T\-O\-R des pieds, servomoteurs, télémètre, caméra, etc.

\paragraph*{Firmware}

Fonctionnalités\-:


\begin{DoxyItemize}
\item Pilotage à partir d'un terminal série standard
\begin{DoxyItemize}
\item Utilisation simple via des commandes type \char`\"{}\-A\-T\char`\"{} définies dans le manuel utilisateur
\end{DoxyItemize}
\item Affichage des informations de contact des pieds
\begin{DoxyItemize}
\item Envoi automatique de l'état sur la la liaison série (commande A\-T)
\end{DoxyItemize}
\item Affichage des informations de la tête chercheuse
\begin{DoxyItemize}
\item À définir
\end{DoxyItemize}
\end{DoxyItemize}

\paragraph*{Contraintes de développement}


\begin{DoxyItemize}
\item Utilisation de git
\item Utilisation de Keil/\-Real\-View
\item Documentation du code avec une syntaxe compatible doxygen
\item Rédaction d'un manuel utilisateur
\end{DoxyItemize}

\subsection*{Découpage du projet}

Le développement est découpé en plusieurs tâches, chacune étant indépendante mais peut en nécessiter d'autres pour fonctionner (contraintes de précédence).

\subparagraph*{Tâche\-: Interface de communication}

Tâche en charge de gérer la liaison série et de transférer les commandes A\-T.


\begin{DoxyItemize}
\item Hardware \-: U\-A\-R\-T
\begin{DoxyItemize}
\item Entrée \-: F\-I\-F\-O (Queue)
\begin{DoxyItemize}
\item Type de donnée \-: structure x\-A\-T\-Command
\end{DoxyItemize}
\item Sortie \-: F\-I\-F\-O (Queue)
\begin{DoxyItemize}
\item Type de donnée \-: structure x\-A\-T\-Command
\end{DoxyItemize}
\end{DoxyItemize}
\end{DoxyItemize}

\subparagraph*{Tâche\-: Télémètre infrarouge}

Tâche en charge de gérer le télémètre infrarouge. Lit la tension de sortie du Sharp pour la convertir en distance.


\begin{DoxyItemize}
\item Hardware \-: A\-D\-C
\item Entrée \-: auncune
\item Sortie \-: distance en cm
\begin{DoxyItemize}
\item Type de donnée \-: entier (8 ou 16 bits)
\end{DoxyItemize}
\end{DoxyItemize}

\subparagraph*{Tâche\-: Contacts T\-O\-R (pieds)}

Tâche en charge de gérer l'état des contacts des pieds. Envoi des commandes A\-T à l'interface de communication lors de changement des états (pas d'interruptions).


\begin{DoxyItemize}
\item Hardware \-: G\-P\-I\-O
\item Entrée \-: auncune
\item Sortie \-: distance en cm
\begin{DoxyItemize}
\item Type de donnée \-: entier (8 ou 16 bits) 
\end{DoxyItemize}
\end{DoxyItemize}