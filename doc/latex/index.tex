Mini-\/projet hexapode

/!\textbackslash{} Projet non fonctionnel

\subsection*{Cahier des charges}

Nous sommes en charge du développement du firmware d'un robot hexapode, basé sur Free\-R\-T\-O\-S. Le robot sera piloté via une liaison série sans fil type Zig\-Bee.

\paragraph*{Mécanique}

Le robot hexapode dispose de 6 pattes en 2 rangées, chaque patte est articulée en trois points via 3 servomoteurs classiques de modélisme. À cela s'ajoute une tourelle 2 axes sur l'avant équipée d'un télémètre infrarouge et d'une caméra C\-M\-U\-C\-A\-M2. De plus, chaque patte possède un contact T\-O\-R au bout du pied.

\paragraph*{Électronique}

La cible est un L\-P\-C17xx auquel sont connectés directement tous les éléments du robot \-: contacts T\-O\-R des pieds, servomoteurs, télémètre, caméra, etc.

\paragraph*{Firmware}

Fonctionnalités\-:


\begin{DoxyItemize}
\item Pilotage à partir d'un terminal série standard
\begin{DoxyItemize}
\item Utilisation simple via des commandes type \char`\"{}\-A\-T\char`\"{} définies dans le manuel utilisateur
\end{DoxyItemize}
\item Affichage des informations de contact des pieds
\begin{DoxyItemize}
\item Envoi automatique de l'état sur la la liaison série (commande A\-T)
\end{DoxyItemize}
\item Affichage des informations de la tête chercheuse
\begin{DoxyItemize}
\item À définir
\end{DoxyItemize}
\end{DoxyItemize}

\paragraph*{Contraintes de développement}


\begin{DoxyItemize}
\item Utilisation de git
\item Utilisation de Keil/\-Real\-View
\item Documentation du code avec une syntaxe compatible doxygen
\item Rédaction d'un manuel utilisateur
\end{DoxyItemize}

\subsection*{Fonctionnement global}

Le système est architecturé autour de tâches gatekeeper gérant le matériel, de tâches de contrôle et un interpréteur. Cet interpréteur est la passerelle entre le robot et le \char`\"{}monde extérieur\char`\"{} et parle \char`\"{}actions\char`\"{} côté robot et \char`\"{}commandes A\-T\char`\"{} de l'autre. Chaque tâche écoute des actions particulières et en émet vers l'interpréteur. Charge à l'interpréteur de renvoyer les actions vers d'autres tâches ou de convertir ces actions en commande A\-T et inversement et si nécessaire. Les tâches gatekeeper n'émettent pas d'actions mais peuvent en écouter (configuration par ex.).

Une action est une structure de type Action qui contient un identifiant d'action destiné à reconnaitre le type de donnée associée. Cet I\-D permet également de transférer l'action aux tâches qui l'écoutent (via des queues Free\-R\-T\-O\-S).

Action \{ Action\-I\-D x\-I\-D; void $\ast$pv\-Data; \}

\subsection*{Interpréteur}

L'interpréteur \-:
\begin{DoxyItemize}
\item converti les commande A\-T en actions
\item écoute les actions émises par les tâches de contrôle (Queue)
\item renvoi les actions reçues vers d'autres tâches qui les écoutent (vers Queues)
\item converti les actions en commandes A\-T si nécessaire
\end{DoxyItemize}

Interface \-: A\-T\-Com\-Com (voir tâches gatekeeper)
\begin{DoxyItemize}
\item Entrée \-: F\-I\-F\-O (Queue)
\begin{DoxyItemize}
\item Type de donnée \-: Action
\end{DoxyItemize}
\item Sortie \-: aucune
\end{DoxyItemize}

\subsubsection*{Tâches gatekeeper}

Les tâches gatekeeper gèrent le matériel. Elle peuvent écouter des actions mais pas en émettre. Chaque tâche met une donnée à disposition des autres tâches de façon sécurisée (mutex, sémaphore ou queue).

\paragraph*{Tâche\-: Interface de communication (A\-T\-Com\-Com \mbox{[}A\-T Commands Communication\mbox{]})}

Tâche en charge de gérer la liaison série et de transférer les commandes A\-T.


\begin{DoxyItemize}
\item Hardware \-: U\-A\-R\-T (L\-P\-C\-Open U\-A\-R\-T)
\item Entrée \-: F\-I\-F\-O (Queue)
\begin{DoxyItemize}
\item Type de donnée \-: structure x\-A\-T\-Command
\end{DoxyItemize}
\item Sortie \-: F\-I\-F\-O (Queue)
\begin{DoxyItemize}
\item Type de donnée \-: structure x\-A\-T\-Command
\end{DoxyItemize}
\end{DoxyItemize}

\paragraph*{Tâche\-: Télémètre infrarouge}

Tâche en charge de gérer le télémètre infrarouge. Lit la tension de sortie du Sharp pour la convertir en distance. La conversion est effectuée à intervalles régulières.


\begin{DoxyItemize}
\item Hardware \-: A\-D\-C (L\-P\-C\-Open A\-D\-C)
\item Entrée \-: auncune
\item Sortie \-: distance en cm
\begin{DoxyItemize}
\item Type de donnée \-: entier (8 ou 16 bits)
\item Sécurité \-: Mutex
\end{DoxyItemize}
\end{DoxyItemize}

\paragraph*{Tâche\-: Contacts T\-O\-R (pieds)}

Tâche en charge de gérer l'état des contacts des pieds. (mode pooling ou via interruptions)


\begin{DoxyItemize}
\item Hardware \-: G\-P\-I\-O (L\-P\-C\-Open G\-P\-I\-O)
\item Entrée \-: aucune
\item Sortie \-: champ de bit
\begin{DoxyItemize}
\item Type de donnée \-: entier (32 bits)
\item Sécurité \-: Mutex
\end{DoxyItemize}
\end{DoxyItemize}

\subsubsection*{Tâches de contrôle}

\paragraph*{Tâche\-: C\-M\-Ucam}

Tâche en charge de gérer la C\-M\-Ucam.


\begin{DoxyItemize}
\item Interface \-: C\-M\-Ucom (interface de comm avec la C\-M\-Ucam)
\item Entrée \-: Action C\-M\-Ucam\-In
\item Sortie \-: Action C\-M\-Ucam\-Out
\end{DoxyItemize}

\paragraph*{Tâche\-: Arm\-Control}

Tâche en charge de gérer les mouvements des pattes de l'hexapode via les servomoteurs (interface de haut niveau).


\begin{DoxyItemize}
\item Interface \-: Servo\-Driver (interface de pilotage des servomoteurs)
\item Entrée \-: Action Arm\-Movement
\item Sortie \-: Action Arm\-Status
\end{DoxyItemize}

\paragraph*{Tâche\-: Head\-Control}

Tâche en charge de gérer les mouvements de la tête supportant la C\-M\-Ucam et le capteur de distance (interface de haut niveau).


\begin{DoxyItemize}
\item Interface \-: Servo\-Driver (interface de pilotage des servomoteurs)
\item Entrée \-: Action Head\-Movement
\item Sortie \-: Action Arm\-Status
\end{DoxyItemize}

\paragraph*{Tâche\-: Movement}

Tâche permettant de contrôler les mouvements de l'hexapode de façon simple (avancer, tourner, etc). Coordonne les mouvements des pattes pour générer des mouvements cohérents.


\begin{DoxyItemize}
\item Interface \-: Arm\-Control
\item Entrée \-: Action Movement
\item Sortie \-: Action Movement\-Status
\end{DoxyItemize}

\subsubsection*{Librairies}

Les librairies sont créées pour contrôler le hardware. Ce ne sont pas des tâches et ne doivent pas en dépendre.

\paragraph*{Librairie \-: contrôle servomoteurs (Servo\-Driver)}

Permet de contrôler les servomoteurs. Le module P\-W\-M1 est utilisé pour générer les 20 signaux avec une période de 21ms et une précision à la microseconde. Voir le fichier \hyperlink{servodriver_8h}{servodriver.\-h} pour les détails sur son fonctionnement.

\paragraph*{Librairie \-: interface T\-O\-R}

La librairie T\-O\-R permet de récupérer l'état des capteurs T\-O\-R des pieds de l'hexapode.

\paragraph*{Librairie \-: interface Sharp}

La librairie I\-R\-\_\-\-Sharp gère le ou les capteurs I\-R Sharp du robot. Elle permet de récupérer la distance en centimètres. /!\textbackslash{} Seul le canal 0 est fonctionnel

\paragraph*{Librairie \-: interface C\-M\-Ucam (C\-M\-Ucom)}

Le portage de la librairie Arduino pour la C\-M\-Ucam4 était prévu, mais devant la masse de travail il a été abandonné. Une méthode pour le portage consiste à effectuer un wrapper de C\-M\-Ucom4 pour L\-P\-C\-Open, mais des problèmes se posent en raison de l'architecture de l'interface U\-A\-R\-T de L\-P\-C\-Open qui ne s'y prête pas vraiment. Autre solution \-: porter C\-M\-Ucam4.\-c/.h pour utiliser L\-P\-C\-Open. 2000 lignes de code à analyser. 